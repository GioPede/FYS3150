\documentclass[10pt,a4paper,titlepage]{article}
\usepackage{latexsym}
\usepackage[utf8x]{inputenc}
\usepackage{booktabs,caption,amsfonts,amssymb,fancyhdr, amsmath}
\usepackage[italian]{babel}
\usepackage{indentfirst}
\usepackage{float}
\renewcommand*{\familydefault}{\sfdefault}
\usepackage{graphicx}
\usepackage{subcaption}
\captionsetup[table]{position=top}
\addtolength{\textwidth}{1cm}
\addtolength{\hoffset}{-1cm}
\pagestyle{headings}
\begin{document}
\begin{center}
{\LARGE \bfseries Computational Physics\par}
\vspace{0.5cm}
{\LARGE \bfseries Project 1 \par}
\end{center}

\vspace{1cm}

\begin{tabular*}{\textwidth}{@{}l@{\extracolsep{\fill}}l@{}}
Academic year 2015-2016	 &Team group: \\
						&Alessio Pizzini\\
                        & Giulio Isacchini\\
                        &Giovanni Pederiva\\
                      &Mattia Alberto Ubertini\\
                       
 
                        
\end{tabular*}
\begin{center}
\hrule height 2 pt
\end{center} 
\paragraph{Sto scrivendo un po di cose molto generali per ora, leggete e in caso aggiungete/corregete}
In this project we've solved a one-dimensional Poisson equation with Dirichlet boundary conditions.
The equation reads: $-u^{''}(x)=f(x)$ with $x [0;1]$ and $u(0)=u(1)=0$.
These kinds of equations can be solved analytically. 
In our case, $f(x)=100e^{-10x}$, so that the previous equation has a solution 
$u_e(x)=1 − (1 − e^{−10} )x − e^ {−10x}$.
In this work we've solved it numerically and then we've compared our solution with the analytic one. 
The first step we did has been the discretization of our domain and consequently of the functions in the equation. Then through the 3-point formula for the second derivative we've rewritten the equation in this way: $-\frac{u_{i-1}+u{i+1}-2u_{i}}{h^2}=f_i$ .
At this point the problem becomes a linear equation system with unknown variables the $u_{i}$, where $u_{i}$ means the function u evaluated at the point $x_{i}$ and the same with $f_{i}$.
The system can be written as $A u = \tilde {f}$ where  $\tilde {f}= h^2 f$. 
The matrix is tridiagonal, and the tridiagonal system related to it can be written as: $$a_iv_{i-1}+b_iv_i+c_iv_{i-1}=\tilde{f_i}$$
in our case the coefficients are costant, precisely $b=2$ , $ a=c=-1$. 
We've coded an alghoritm for solving this systems. 
The alghoritm written is built upon the Gaussian elimination method. 
The alghoritm consists in a procedure divided in two parts, the first step basically let us to reduce the matrix and consecutively to compute directly the last component $u_i$, in the second part we use the last component to compute recursively the previous ones.  


\section{A}
In order to solve the differential equation:
\[
-u''(x) = f(x), \hspace{0.5cm} x\in(0,1), \hspace{0.5cm} u(0) = u(1) = 0.
\]
you discretize the problem into the numerical equation
\[
   -\frac{v_{i+1}+v_{i-1}-2v_i}{h^2} = f_i  \hspace{0.5cm} \mathrm{for} \hspace{0.1cm} i=1,\dots, n,
\]
where $f_i=f(x_i)$.
If you define the three quantities:
\[
    {\bf A} = \left(\begin{array}{cccccc}
                           2& -1& 0 &\dots   & \dots &0 \\
                           -1 & 2 & -1 &0 &\dots &\dots \\
                           0&-1 &2 & -1 & 0 & \dots \\
                           0& \dots   & \dots &\dots   &\dots & \dots \\
                           0&\dots   &  &-1 &2& -1 \\
                           0&\dots    &  & 0  &-1 & 2 \\
                      \end{array} \right)
	\hspace{0.5cm}
    {\bf v} = \left(\begin{array}{c}
                           v_1\\
                           v_2\\
                           \dots\\
                           v_i\\
                           \dots\\
                           v_n\\
                      \end{array} \right)
                      \hspace{0.5cm}
	{\bf \tilde{g}} = \left(\begin{array}{c}
                           \tilde{g}_1\\
                           \tilde{g}_2\\
                           \dots\\
                           \tilde{g}_i\\
                           \dots\\
                           \tilde{g}_n\\
                      \end{array} \right)                  
\]
with $\tilde{g}_i=h^2 f_i$
it is trivial to show that the numerical equation can be rewritten as:
\[
   {\bf A}{\bf v} = \tilde{{\bf b}},
\]
In order to solve it with compare three different algorithms:
\begin{itemize}
\item {\bf Tridiagonal Matrix Algorithm}, also known as the Thomas algorithm
\item Standard Gaussian Elimination
\item LU decomposition
\end{itemize}
\paragraph{Tridiagonal Matriix Algorithm} Given a general tridiagonal matrix system:
\[
    {\bf M} = \left(\begin{array}{cccccc}
                           b_1& c_1& 0 &\dots   & \dots &0 \\
                           a_2 & b_2 & c_3 &0 &\dots &\dots \\
                           0&a_3 &2 & b_3 & 0 & \dots \\
                           0& \dots   & \dots &\dots   &\dots & 0 \\
                           0&\dots   &\dots  &\dots  &\dots& c_{n-1} \\
                           0&\dots &\dots  &0  &a_{n} & b_n \\
                      \end{array} \right)
                       \left(\begin{array}{c}
                           v_1\\
                           v_2\\
                           \dots\\
                           v_i\\
                           \dots\\
                           v_n\\
                      \end{array} \right)
	= \left(\begin{array}{c}
                           \tilde{g}_1\\
                           \tilde{g}_2\\
                           \dots\\
                           \tilde{g}_i\\
                           \dots\\
                           \tilde{g}_n\\
                      \end{array} \right)    
\]   
The algorithm can be summed up in:
\begin{itemize} 
\item Forward sweep:
\[\huge
	c'_i= \begin{cases} 
			\frac{c_i }{b_i }  & i=1 \\
			\frac{c_i}{b_i-a_i c'_{i-1}} & i=2,\dots,n. 
		\end{cases}
	\hspace{0.5cm}
	\tilde{g}'_i= \begin{cases} 
					\frac{\tilde{g}_i}{b_i}  & i=1 \\
					\frac{\tilde{g}_i- a_i \tilde{g}'_{i-1}}{b_i-a_i c'_{i-1}} & i=2,\dots,n. 
				\end{cases}
\]
\end{itemize}
\end{document}